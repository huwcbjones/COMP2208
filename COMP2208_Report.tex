\documentclass[a4paper]{article}

\usepackage[utf8]{inputenc}
\usepackage[margin=1in]{geometry}
\usepackage[bookmarks]{hyperref}
\usepackage{graphicx}
\usepackage{listings}
\usepackage{color}
\definecolor{col_light_grey}{rgb}{0.9, 0.9, 0.9}
\definecolor{col_green}{rgb}{0,0.6,0}
\definecolor{col_grey}{rgb}{0.5,0.5,0.5}
\definecolor{col_mauve}{rgb}{0.58,0,0.82}

\lstset{
  backgroundcolor=\color{col_light_grey},
  basicstyle=\footnotesize,
  breaklines=true,
  captionpos=b,
  commentstyle=\color{col_green},
  escapeinside={\%*}{*)},
  keywordstyle=\color{blue},
  stringstyle=\color{col_mauve},
  tabsize=2,
}
\lstset{language=java}

\hypersetup{
  pdfinfo={
    Title={COMP2208 Assignment},
    Author={Huw Jones},
  },
  colorlinks=false,
  pdfborder=0 0 0,
}

\pagestyle{headings}

\author{Huw Jones \\27618153}
\title{COMP2208 Assignment}

\begin{document}
\maketitle
\newpage

\section{Approach}
In order to analyse the differences in scalability, I decided to build a framework that would allow me to minimise the time spent on writing code.
At the moment, I am most familiar with Java, therefore that is the language I chose to build my solution in.
My code is nowhere near ``good'' or optimised (in terms of real time running, not nodes expanded), but it works.

\section{Evidence}

\section{Scalability}

\section{Extras \& Limitations}

\section{References}

\section{Code}

\end{document}
